\section{Beamstop: A neutron absorbing area}
\label{beamstop}
\index{Optics!Beam stop}

%\component{Beamstop}{System}{$x_{min}$, $x_{max}$, $y_{min}$,
%$y_{max}$}{$r$}{}
\mcdoccomp{optics/Beamstop.parms}

The component {\bf Beamstop} can be seen as the reverse of
the {\bf Slit} component.
It sets up an area at the $z=0$ plane, and propagates the neutrons
onto this plane (by the kernel call PROP\_Z0).
Neutrons within this area are ABSORB'ed,
while all other neutrons are unaffected.

By using this component, some neutrons contributing to the background
in a real experiment will be neglected.
These are the ones that scatter off the side
of the beamstop, or penetrates the absorbing material.
Further, the holder of the beamstop is not simulated.

{\bf Beamstop} can be either circular or rectangular.
The input parameters of {\bf Beamstop} are the four coordinates,
$(x_{\rm min}, x_{\rm max}, y_{\rm min}, y_{\rm max})$
defining the opening of a rectangle, or the radius $r$ of
a circle, depending on which parameters are specified.

If the "direct beam" (e.g. after a monochromator or sample) should not be
simulated, it is possible to emulate an ideal beamstop 
so that only the scattered beam is left;
without the use of {\bf Beamstop}:
This method is useful for instance in the case where only neutrons 
scattered from a sample are of interest. 
The example below removes the direct beam and 
any background signal from other parts of the instrument
\begin{lstlisting}
COMPONENT MySample=V_sample(...) AT (...)
EXTEND
%{
  if (!SCATTERED) ABSORB;
%}
\end{lstlisting}
