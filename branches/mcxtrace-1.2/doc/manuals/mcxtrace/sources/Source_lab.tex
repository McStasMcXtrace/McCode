\section{Source\_lab: X-ray tube laboratory source}
\label{s:source-lab}
\index{Sources!X-ray tube laboratory source}

\mcdoccomp{sources/Source_lab.parms}

\textbf{Source\_lab} is a model of a laboratory X-ray tube. An electron ray hits a
target of specified material. Currently, only single material targets are
allowed. To model multiple material targets one could construct a model with two
or more sources simultaneously. This has consequences for intensity of the source which should be downscaled accordingly.

An electron beam of rectangular transverse crossection (\textit{width,height}) and energy \textit{E0}
impinges on the target of material. Wrt. the electron beam, the target is
considereded infinitely thick. The beam is considered to have uniform
intenisty. Thus, the spatial distribution of x-ray generation will be
exponential in the depth of the material.

Further, an exit aperture is defined with dimensions
(\textit{focus\_xw,focus\_yh}). The centre of the aperture is situated at a
distance \textit{dist} \si{m} from where the electron beam hits the target slab
at an elevation of \textit{take\_off} (see Figure~\ref{f:source_lab}).  

The \textbf{Source\_lab} coordinate system has its origin in the center of the
elctron beam at the surface of the
anode material and is oriented such that the z-axis points at the center of the
exit window, and the x-axis is parallel to the \emph{width} of the electron
beam. 

Note that the exit aperture is merely an opening. If the
material absorption of a window, e.g. Be, is to be taken into account a
\textbf{Filter} (section~\ref{s:filter}) should be inserted after the exit
aperture. 

\begin{figure}
\label{f:source_lab}
\centering
\def\svgwidth{\columnwidth}
%\section{Source\_lab: X-ray tube laboratory source}
\label{s:source-div}
\index{Sources!X-ray tube laboratory source}

\component{Source\_lab}{System}{ $w$, $h$, $\delta_h$, $\delta_v$, $E_0$, $\Delta E$}{$\lambda_0$, $\Delta\lambda$, gauss}{Validated. t=0}

{\bf Source\_lab} is a model of a laboratory X-ray tube. An electron ray hits a
target of specified material. Currently, only single materiual targets are
allowed\footnote{To model multiple material targets one could construct a model with two
or more sources simultaneously. This has consequences for intensity of the source which should be downscaled accordingly.}.

An electron beam of transverse crossection ($x_0,z_0$) and energy $E_0$
impinges on the target of material. Wrt. the electron beam, the target is
considereded infinitely thick. The beam is considered to have uniform
intenisty. Thus, the spatial distribution of x-ray generation will be
exponential in the depth of the material.

Further, an exit aperture is defined with dimensions ($x_{width},y_{height}$). The centre of the aperture is situated at a distance $wd$ m from where the electron beam hits the target slab at an elevation of $take\_off$ (see Figure~\ref{f:source_lab}).  
Note that the center of the exit aperture is the reference point of the
\verb+Source\_lab+ coordinate system. In other words, the position specified in
the instrument file \verb+AT (x,y,z) RELATIVE somewhere+ is the center of the
exit aperture. Also note that the exit aperture is merely an opening. If the material absorption of the window, e.g. Be, is to be taken into account a \verb+Filter.comp+ (section~\ref{s:filter}) could be inserted after the exit aperture. 

\begin{figure}
\label{f:source_lab}
\caption{Geometry of the \texttt{Source\_lab} component} 
\end{figure}

For each photon to be generated, a monte carlo choice is made to either
generate either a Bremstrahlung photon or one from one of the x-ray emission
lines of the material. $frac$ of the photons are generated from characteristic
emission, and $1-frac$ from Bremsstrahlung. In most cases Bremstrahlung is
unwanted background, which is why the default is $0.9$. Note that this
\emph{only} governs how much of the available statistics is diverted into
simulating backgrouns. It does not have an impact on what intensity is detected
in subsequent monitors --- only on the errorbars of the detected numbers.

The spectral characteristics of the generated Bremsstrahlung is goverened by
the model suggested by Kramer~\cite{kramer_23}. Although disputed in several
subsequent papers, the model is simple, and sufficiently accurate for many
background estimation purposes.

Characteristic emission on the other hand is sampled from a set of Lorentzian
functions with central wavelengths found in the work by \cite{bearden} with
spectral widths taken from \cite{deutsch}.

An example of beam spectral characteristics emitted from a Cu-anode targate detected $1$ mm  from an exit aperture of $1\times 1$ cm $10$ cm fround the target at a $take\_off$ angle of $6^\circ$. is seen in figure~\ref{f:source_lab_spectrum}.
\begin{figure}
\label{f:source_lab_spectrum}
\caption{Intensity vs. wavlenghth for a Cu-anode laboratory source.}
\end{figure}

\import{figures/}{source_lab.pdf_tex}
\caption{Geometry of the \textbf{Source\_lab} component. A electron beam impinges at a right angle on
an anode material, where X-rays are generated. The Origin is defined to be a the centre of the electron beam
on the anode surface, and the coodrinate system is oriented such that the $\boldsymbol{z}$-axis point towards
the exit aperture.} 
\end{figure}

For each photon to be generated, a Monte Carlo choice is made to either
generate either a Bremstrahlung photon or one from one of the x-ray emission
lines of the material. $\mathit{frac}$ of the photons are generated from characteristic
emission, and $1-\mathit{frac}$ from Bremsstrahlung. In most cases Bremstrahlung is
unwanted background, which is why the default is $0.9$. Note that this
\emph{only} governs how much of the available statistics is diverted into
simulating Bremsstrahlung. It does not have an impact on what intensity is detected
in subsequent monitors --- only on the errorbars of the detected numbers.

The spectral characteristics of the generated Bremsstrahlung is goverened by
the model suggested by Kramer~\cite{kramers1923}. Although disputed in several
subsequent papers, the model is simple, and sufficiently accurate for many
background estimation purposes.

Characteristic emission on the other hand is sampled from a set of Lorentzian
functions with central wavelengths found in the work by \cite{bearden1967x} with
spectral widths taken from \cite{krause1979natural}.

An example of beam spectral characteristics emitted from a Cu-anode target
detected $1$ mm  from an exit aperture of $1\times 1$ \si{cm\squared} $10$ \si{cm} downstream from the
target at a \textit{take\_off}angle of $6\si{\degree}$ is seen in
figure~\ref{f:source_lab_spectrum}.
\begin{figure}
\label{f:source_lab_spectrum}
\caption{Intensity vs. wavelenghth for a Cu-anode laboratory source.}
\end{figure}
