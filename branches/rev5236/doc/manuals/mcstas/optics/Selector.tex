\section{Selector: another approach to describe a rotating velocity selector}
\label{selector}
\index{Optics!Velocity selector}

%\component{Selector}{System}{$xmin$, $xmax$, $ymin$, $ymax$, $len$, $num$, $width$, $radius$, $\alpha$, $feq$}{}{validated, position is center of input aperture}
\mcdoccomp{optics/Selector.parms}

The component {\bf Selector} describes the same kind of rotating velocity selector as {\bf V\_selector} - compare
description there - but it uses different parameters and a different algorithm:

The position of the apertures relative to the z-axis (usually the beam centre) is defined by the four parameters
$xmin, xmax, ymin, ymax$. Entry and exit apertures are always identical and situated directly before and behind
the rotor.
There are $num$ blades of thickness $width$ twisted by the angle $\alpha$ (in degrees) on a length $len$.
The selector rotates with a speed $feq$ (in rotation per second); its axle is in a distance $radius$ below the z-axis.

First the neutron is propagated to the entrance window. The loss of neutrons hitting the thin side
of the blades is taken into account by multiplying the neutron weight by a factor

\begin{equation}
   p(r) = \theta_i(r) / \theta_o
\end{equation}

\begin{equation}
   \theta_o = 360^o / num
\end{equation}

$\theta_i$ is the opening between two blades for the distance $r$ between the neutron position (at the entrance)
and the selector axle. The difference between $\theta_o$ and $\theta_i$ is determined by the blade thickness.
The neutron is now propagated to the exit window. If it is outside the regarded channel (between the two actual
blades), it is lost; otherwise it remains in the exit plane.

WARNING - Differences between {\bf Selector} and {\bf V\_selector}:
\begin{itemize}
\item {\bf Selector} has a different coordinate system than {\bf V\_selector};
in {\bf Selector} the origin lies in the entrance plane of the selector.
\item The blades are twisted to the other side, i.e. to the left above the axle in {\bf Selector}.
\item Speed of rotation is given in rotation per second, not in rotations per minute as in {\bf V\_selector}.
\end{itemize}



