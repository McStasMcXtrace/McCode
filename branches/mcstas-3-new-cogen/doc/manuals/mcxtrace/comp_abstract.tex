\chapter{The component library: Abstract}
\label{s:components}
\index{Library!Components|textbf}

This chapter presents an abstract of existing components.
As a complement to this chapter and the
detailed description in the \MCX component manual,
you may use the \verb+mxdoc -s+ command to obtain the on-line
component documentation and refer to the McXtrace web-page~\cite{mcxtrace_webpage}
where all components are documented using the MxDoc system.

\section{A short overview of the \MCX\ component library}
\label{s:comp-overview}

The table in this section gives a quick overview of available \MCX\ components
provided with the distribution, in the \verb+MCXTAS+ library. The
location of this library is detailed in section~\ref{s:files}.
All of them are believed to be reliable, and some amount of systematic
tests have been carried out.
However, no absolute guaranty may be given concerning their accuracy.\index{Environment variable!MCXTAS}

The \verb+contrib+ directory of the library contains components
that were submitted by \MCX users,
but where responsibility has not (yet) been taken by the \MCX core team. \index{Library!Components!contrib}

%Additionally the \verb+obsolete+ directory of the library gathers components that were renamed, or considered to be outdated.
%These component are kept for backwards compatibility and
%they still all work as before.\index{Library!Components!obsolete}

The \verb+mxdoc+ front-end (section~\ref{s:mxdoc-run}) enables to display both the
catalog of the \MCX\ library, e.g using: \index{Tools!mxdoc}
\begin{quote}
  \verb|mxdoc|
\end{quote}
as well as the documentation of specific components, e.g with:
\begin{quote}
  \verb|mxdoc --text| \textit{name} \\
  \verb|mxdoc| \textit{file.comp}
\end{quote}
The first line will search for all components matching the \textit{name}, and display their help section as text, where as the second example will display the help corresponding to the \textit{file.comp} component, using your BROWSER\index{Environment variable!BROWSER} setting, or as text if unset. The \verb+--help+ option will display the command help, as usual.

\begin{table}
  \begin{center}
    {\let\my=\\
    \begin{tabular}{|p{0.32\textwidth}|p{0.62\textwidth}|}
      \hline
       {\bfseries \MCX/sources} & Description \\
       \hline
Source\_pt & Point source with uniform, Gaussian or general wavelength/energy distribution.\\
Source\_flat & Flat surface source with uniform, Gaussian or general wavelength/energy distribution.\\
Source\_div & Flat surface source with general wavelength/energy distribution and a given divergence.\\
Source\_gaussian & Flat source with a Gaussian cross section and a specified divergence.\\
Source\_lab & Full featured model of a laboratory X-ray source.\\
      \hline
    \end{tabular}
    \caption{Source and source-related components of the \MCX\ library.}
    \label{t:comp-sources}
    \index{Library!Components!sources}
    }
  \end{center}
\end{table}


\begin{table}
  \begin{center}
    {\let\my=\\
    \begin{tabular}{|p{0.32\textwidth}|p{0.62\textwidth}|}
      \hline
       {\bfseries \MCX/optics} & Description \\
       \hline
 Arm                &  Arm/optical bench. \\
 Beamstop          &   Rectangular/circular beam stop. \\
 Filter        &   A general absorption filterm which can be any material. \\
 Lens\_simple & Model of a thin compound refractive lens\\
 Lens\_parab  & Detailed model of a parabolic compound refractive lens of any material\\
 Mirror\_curved & Cylindrically curved mirror with scalar reflectivity\\
 Mirror\_ellioptic & Elliptically curved mirror with scalar reflectivity\\
 Mirror\_parabolic & Parabolically curved mirror with scalar reflectivity\\
 Multilayer\_elliptic & General, full featured model of an elliptically curved multilayer mirror\\
 Slit & Perfect slit, may be either rectangular or circular\\
 Slit\_N & Multichanneled slit, i.e. a grating\\
        \hline
\end{tabular}
    \caption{Optics components of the \MCX library.}
    \label{t:comp-optics}
    \index{Library!Components!optics}
    }
  \end{center}
\end{table}

\begin{table}
  \begin{center}
    {\let\my=\\
    \begin{tabular}{|p{0.32\textwidth}|p{0.62\textwidth}|}
      \hline
       {\bfseries \MCX/samples} & Description \\
       \hline
  PowderN      &  General powder sample with N
                scattering vectors, using a data file. Can assume \emph{concentric} shape,
		i.e. can be used to model sample enviroment.\\
  
  Perfect\_crystal & Model of a prefect crystal, for instance to be used as a monochromator\\ 
  Saxs\_spheres  & Simple sample for Small Angle X-ray Scattering - hard spheres \\
  Single\_crystal & Mosaic single crystal with multiple scattering vectors
                    using a data file. \\
        \hline
                  \end{tabular}
    \caption{Sample components of the \MCX library.}
    \label{t:comp-samples}
    \index{Library!Components!samples}
    }
  \end{center}
\end{table}

\begin{table}
  \begin{center}
    {\let\my=\\
    \begin{tabular}{|p{0.32\textwidth}|p{0.62\textwidth}|}
      \hline
       {\bfseries McXtrace/monitors} & Description \\
       \hline
%DivLambda\_monitor &  Divergence/wavelength monitor. \\
DivPos\_monitor  &    Divergence/position monitor (acceptance diagram). \\
Divergence\_monitor &  Horizontal+vertical
                    divergence monitor (2D). \\
%EPSD\_monitor    &    A monitor measuring neutron intensity vs. position, x,
%                    and neutron energy, E. \\
E\_monitor       &    Energy-sensitive monitor. \\
%Hdiv\_monitor    &    Horizontal divergence monitor \\
L\_monitor        &  Wavelength-sensitive monitor. \\
Monitor          &   Simple single detector/monitor. \\
%Monitor\_4PI     &    Monitor that detects ALL non-absorbed neutron rays. \\
%Monitor\_nD      &   General monitor that can output
%                    0/1/2D signals (Intensity or signal vs. [something]
%                    and vs. [something] ...). \\
PSD\_monitor     &    Position-sensitive monitor. \\
PSD\_monitor\_4PI  &   Spherical position-sensitive detector. \\
%PSDcyl\_monitor  &    A 2D position-sensitive monitor. The shape is
%                    cylindrical with the axis vertical. The monitor
%                    covers the whole cylinder (360 degrees). \\
%PSDlin\_monitor   &   Rectangular 1D PSD, measuring intensity vs.
%                    vertical position, x. \\
PreMonitor\_nD    &   This component is a PreMonitor that is to be
                    used with one Monitor\_nD,
                    in order to record some photon parameter correlations. \\
TOFLambda\_monitor &  Time-of-flight vs. wavelength monitor. \\
%TOF\_cylPSD\_monitor & Cylindrical (2pi) PSD Time-of-flight monitor. \\
TOF\_monitor     &    Rectangular Time-of-flight monitor. \\
%TOFlog\_mon      &    Rectangular Time-of-flight monitor with logarithmic
%                    time binning. \\
      \hline
    \end{tabular}
    \caption{Selected Monitor components of the \MCX\ library.}
    \label{t:comp-monitors}
    \index{Library!Components!monitors}
    }
  \end{center}
\end{table}

\begin{table}
  \begin{center}
    {\let\my=\\
    \begin{tabular}{|p{0.32\textwidth}|p{0.62\textwidth}|}
      \hline
       {\bfseries \MCX/misc} & Description \\
       \hline
 Progress\_bar     &  Displays status of a running simulation.
                      May also trigger intermediate SAVE.\\
 Beam\_spy         & A monitor that displays mean statistics (no output file). \\
      \hline
    \end{tabular}
    \caption{Miscellaneous components of the \MCX\ library.}
    \label{t:comp-misc}
    \index{Library!Components!misc}
    }
  \end{center}
\end{table}

%\begin{table}
%  \begin{center}
%    {\let\my=\\
%    \begin{tabular}{|p{0.24\textwidth}|p{0.7\textwidth}|}
%      \hline
%       {\bfseries \MCX\/contrib} & Description \\
%       \hline
%
%      \hline
%    \end{tabular}
%    \caption{Contributed components of the \MCX\ library.}
%    \index{Library!Components!contrib}
%    \label{t:comp-contrib}
%    }
%  \end{center}
%\end{table}

%\begin{table}
%  \begin{center}
%    {\let\my=\\
%    \begin{tabular}{|p{0.24\textwidth}|p{0.7\textwidth}|}
%      \hline
%       {\bfseries MCXTAS/share} & Description \\
%       \hline
%       adapt\_tree-lib  & Handles a simulation optimisation space for
%       adatative importance sampling.
%                          Used by the Source\_adapt component. \\
%       {\bfseries mcstas-r}      &   Main Run-time library (always included). \\
%       monitor\_nd-lib & Handles multiple monitor types.
%                        Used by Monitor\_nD, Res\_monitor, \ldots \\
%       read\_table-lib  & Enables to read a data table (text/binary) to be used within
%                          an instrument or a component. \\
%       vitess-lib &     Enables to read/write Vitess event binary files.
%                        Used by Vitess\_input and Vitess\_output \\
%      \hline
%    \end{tabular}
%    \caption{Shared libraries of the \MCX library. See Appendix~\ref{c:kernelcalls} for details.}
%    \label{t:comp-share}
%    \index{Library!Components!share}
%    \index{Library!Run-time}
%    }
%  \end{center}
%\end{table}
%
\begin{table}
  \begin{center}
    {\let\my=\\
    \begin{tabular}{|p{0.32\textwidth}|p{0.62\textwidth}|}
      \hline
       {\bfseries \MCX/data} & Description \\
       \hline
 Al.txt, Al.txt, Be.txt, Cu.txt, Fe.txt, I.txt, Mo.txt, Si.txt, Sn.txt & Data files from the NIST-database to be used for absorption and refraction.\\
 get\_xray\_db\_data & script to access NIST-database.\\  
\hline
    \end{tabular}
    \caption{Data files of the \MCX library.}
    \label{t:comp-data}
    \index{Library!Components!data}
    }
  \end{center}
\end{table}

\begin{table}
  \begin{center}
    {\let\my=\\
    \begin{tabular}{|p{0.32\textwidth}|p{0.62\textwidth}|}
      \hline
       {\bfseries \MCX/examples} & Description \\
      \hline
      *.instr & This directory contains example instruments, accessible throught the \verb+mxgui+ ``X-ray site'' menu. \\
      \hline
    \end{tabular}
    \caption{Instrument example files of the \MCX library.}
    \label{t:comp-instr}
    \index{Library!Instruments}
    }
  \end{center}
\end{table}
