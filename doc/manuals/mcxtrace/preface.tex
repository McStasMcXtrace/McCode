% Emacs settings: -*-mode: latex; TeX-master: "manual.tex"; -*-

\addcontentsline{toc}{chapter}{\protect\numberline{}{Preface and acknowledgements}}
\chapter*{Preface and acknowledgements}
This document contains information on the Monte Carlo
x-ray tracing program \MCX version \version, building on the initial
release in October 1998 of the neutron ray tracing program \MCS version 1.0 as presented in Ref.~\cite{nn_10_20}. The reader of this
document is expected to have some knowledge of x-ray and/or neutron scattering,
whereas only little knowledge about simulation techniques is
required. In a few places, we also assume familiarity with the
use of the C programming language and UNIX/Linux.

Support has been kindly given by SAXLAB Aps. through its CEO Karsten Joensen as well 
the ESRF and MAX IV Laboratory. We acknowledge all contributing parties. 

In case of errors, questions, or suggestions, please do not hesitate to
contact the team and the community by either writing to the user mailing list \verb+mcxtrace-users@mcxtrace.org+, 
consulting the \MCX home page~\cite{mcxtrace_webpage}, or leaving a note on the \MCX facebook page \url{https://www.facebook.com/McXtrace}.
A special bug/request reporting service is available \cite{mczilla_webpage}.

If you {\bfseries appreciate} this software, please subscribe to the \verb+mcxtrace-users@mcxtrace.org+ email list, send us a smiley message, and contribute to the package. 

We encourage you to refer to this software when publishing result with the following citation:
E. B. Knudsen, et. al., \textit{Journal of Applied Crystallography}, vol. 46, 2013.

Third party software included in  the distribution \MCX is:
\begin{itemize}
\item perl Math::Amoeba from John A.R. Williams \verb+J.A.R.Williams@aston.ac.uk+.
\item perl Tk::Codetext from Hans Jeuken \verb+haje@toneel.demon.nl+.
\item and optionally PGPLOT from Tim Pearson \verb+tjp@astro.caltech.edu+.
\end{itemize}

The \MCX project was initially supported by Det Strategiske Forskningsråd through the NaBiIT programme. Partners in this joint venture were: 
\begin{itemize}
\item Materials Research Division, Risø DTU, Roskilde, Denmark.
\item Niels Bohr Institute, University of Copenhagen, Denmark.
\item Faculty of Life Sciences, University of Copenhagen, Denmark.
\item SAXLAB ApS. Lundtofte, Denmark.
\item ESRF, Grenoble, France.
\end{itemize}
