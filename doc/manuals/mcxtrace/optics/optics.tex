% Emacs settings: -*-mode: latex; TeX-master: "manual.tex"; -*-

\chapter{Beam optical components:
Arms, slits, filters etc.}
This chapter contains a number of optical components
used to modify the x-ray beam in various ways,
as well as the ``generic'' component \textbf{Arm}.
\index{Library!Components!optics}
\index{Optics|textbf}

\section{Arm: The generic component}
\label{s:arm}
\index{Optics!Point in space (Arm, Optical bench, Coordinate system)}
\mcdoccomp{optics/Arm.parms}

The component \textbf{Arm} is empty; is resembles an optical bench
and has no effect on the x-ray.
The purpose of this component is only to provide a standard
means of defining a local coordinate system within the instrument definition.
Other components may then be
positioned relative to the \textbf{Arm} component
using the \MCX\ meta-language.
The use of \textbf{Arm} components in beamline definitions
is not required but is recommended for clarity.
\textbf{Arm} has no input parameters.

The first Arm instance in an instrument definition may be changed into a
\textbf{Progress\_bar} (sec.~\ref{s:progress-bar}) component in order to display
simulation progress on the fly, and possibly save intermediate results.


\section{Slit: A beam defining diaphragm}
\label{slit}
\index{Optics!Slit}

%\component{Slit}{System}{$x_\textrm{min}$, $x_\textrm{max}$, $y_\textrm{min}$, $y_\textrm{max}$}{$r$, $p_\textrm{cut}$}{}
\mcdoccomp{optics/Slit.parms}

The component \textbf{Slit} is a very simple construction.
It sets up an opening at $z=0$, and propagates the neutrons
onto this plane (by the kernel call PROP\_Z0).
Neutrons within the slit opening are unaffected,
while all other neutrons
are discarded by the kernel call ABSORB.

By using \textrm{Slit}, some neutrons contributing to the background
in a real experiment will be neglected.
These are the ones that scatter off the inner side
of the slit, penetrates the slit material,
or clear the outer edges of the slit.

The input parameters of \textbf{Slit} are the four coordinates,
$(x_\textrm{min}, x_\textrm{max}, y_\textrm{min}, y_\textrm{max})$
defining the opening of the rectangle, or the radius $r$ of
a circular opening, depending on which parameters are specified.

The slit component can also be used to discard insignificant 
({\em i.e.}\ very low weight)
neutron rays, that in some simulations may be very abundant and therefore
time consuming. If the optional parameter $p_\textrm{cut}$ is set, all
neutron rays with $p<p_\textrm{cut}$ are ABSORB'ed.
This use is recommended in connection with \textbf{Virtual\_output}.




\section{Slit\_N: multiple slits}
\label{s:slit-n}
\mcdoccomp{optics/Slit_N.parms}

Documentation pending.


\section{Beamstop: A photon absorbing area}
\label{beamstop}
\index{Optics!Beam stop}

\mcdoccomp{optics/Beamstop.parms}

The component \textbf{Beamstop} can be seen as the reverse of
the \textbf{Slit} component.
It sets up an area at the $z=0$ plane. Photons that hit the plane 
within this area are ABSORB'ed, while all others are unaffected.

By using this component, some photons contributing to the background
in a real experiment will be neglected.
These are the ones that scatter off the side
of the (real) beamstop, or penetrate the absorbing material.
Further, the holder of the beamstop is not simulated.

\textbf{Beamstop} can be either circular or rectangular.
The input parameters of \textbf{Beamstop} are either height and width \textit{(xwidth, yheight)} or the four coordinates,\textit{(xmin, xmax, ymin, ymax)}
defining the opening of a rectangle, or the \textit{radius} of
a circle, depending on which parameters are specified.

If the "direct beam" (e.g. after a monochromator or sample) should not be
simulated, it is possible to emulate an ideal beamstop 
so that only the scattered beam is left;
without the use of \textbf{Beamstop}:
This method is useful for instance in the case where only photons 
scattered from a sample are of interest. 
The example below removes the direct beam and 
any background signal from other parts of the beamline
\begin{verbatim}
COMPONENT MySample=PowderN(...) AT (...)
EXTEND
%{
  if (!SCATTERED) ABSORB;
%}
\end{verbatim}


\section{Filter: A general absorption filter model}
\label{s:filter}
\index{Optics!Filter}
\mcdoccomp{optics/Filter.parms}

This component is a filter in the shape of a rectangular block or a general
shape defined by a set of polygons. Given an input file containing material
parameters. Neccessary parameters are nominal density and a parameterization of
the linear attenuation coefficient, $\mu$ as a function of wavelength (or
energy).

The model is very simple: Firstly the X-ray is traced to find intersection points between ray and filter (0 or 2).
If no intersection is found the x-ray is left untouched and nothing further happens.
Assuming the ray intersects the filter: Secondly, the path length d$l$ within the filter is computed.
Thirdly a $\mu = f(\lambda,\mathrm{material})$ is computed by interpolating in
a datafile, and the x-ray weight is adjusted according to
$p=p\exp(-\mathrm{d}l*\mu)$. The x-ray is left at the point where it exits the
filter block (the $2$nd intersection).

Example data files corresponding to all elements up to $Z=92$ are distributed with \MCX in the
\verb+MCXTRACE/data+ directory as \verb+*.txt+ files. These tables have been
extracted from the NIST FFAST~\cite{NIST-ffast} x-ray database.
To generate other datafiles from the same source a simple shell script:
\verb+MCXTRACE/data/get_xray_db_data+ is also distributed with \MCX
Running this script will connect to the NIST webiste and download a
\verb+.html+ file. This output must now be modified such that \verb+html+-tags
are removed and all header lines begin with $\#$

\subsection{Example}
\label{getNISTdata}
This is an example of how to download and generate datafiles for the \verb+Filter.comp+ and others.

The distributed tables have been extracted from the NIST x-ray database. To ease generation of more dtafiles
from the same source a simple shell script:\\
\verb+MCXTRACE/data/get_xray_db_data+\\
is also distributed with \MCX

Running this script will connect to the NIST webiste and download a \verb+.html+ file. This output must now be modified wuch that \verb+html+-tags
are removed and all header lines begin with $\#$.

\begin {verbatim}
 /usr/local/lib/mcxtrace/data/get_xray_db_data 3 output.dat
\end{verbatim}
where the second parameter (3) is the atom number of the material, for which we want to generate a datafile.
Now open the generated datafile (\verb+output.dat+) with your favourite text editor and make sure the file ends up looking like this
\tiny
\begin{verbatim}
#Li (Z 3)
#Atomic weight: A[r]  6.941000
#Nominal density: rho 5.3300E-01
#    rho[a](barns/atom) = [mu/rho](cm^2 g^-1)  x  1.15258E+01
#    E(eV) [mu/rho](cm^2 g^-1) = f[2](e atom^-1)  x  6.06257E+06
#    2 edges. Edge energies (keV):
#
#
#    K      5.47500E-02  L I    5.34000E-03
#
#Relativistic correction estimate f[rel] (H82,3/5CL) = -9.8613E-04,
#    -6.0000E-04 e atom^-1
#    Nuclear Thomson correction f[NT] = -7.1131E-04 e atom^-1
#
#-------------------------------------------------------------------------------
#Form Factors, Attenuation and Scattering Cross-sections
#Z=3, E = 0.001 - 433 keV
#
#    E        f[1]         f[2]        [mu/rho]    [sigma/rho]  [mu/rho]   [mu/rho][K] lambda
#                                    Photoelectric Coh+inc      Total
#   keV        e atom^-1    e atom^-1   cm^2 g^-1   cm^2 g^-1   cm^2 g^-1   cm^2 g^-1  nm
5.233200E-03  9.08733E-01  0.0000E+00  0.0000E+00  2.3914E-07  2.3914E-07  0.000E+00  2.369E+02
5.313300E-03  8.59283E-01  0.0000E+00  0.0000E+00  2.5404E-07  2.5404E-07  0.000E+00  2.333E+02
5.334660E-03  8.03599E-01  0.0000E+00  0.0000E+00  2.5813E-07  2.5813E-07  0.000E+00  2.324E+02
5.366700E-03  8.56971E-01  1.0769E-01  1.2165E+05  2.6435E-07  1.2165E+05  0.000E+00  2.310E+02
.
.
.
3.788588E+02  3.00000E+00  3.9121E-08  6.2602E-07  8.4389E-02  8.4390E-02  6.123E-07  3.273E-03
4.050001E+02  3.00000E+00  3.3438E-08  5.0054E-07  8.2127E-02  8.2128E-02  4.895E-07  3.061E-03
4.329451E+02  3.00000E+00  2.8581E-08  4.0022E-07  7.9892E-02  7.9892E-02  3.913E-07  2.864E-03
\end{verbatim}
\normalsize
Please make sure you don't forget to remove the html-tags in the bottom of the file as well. In the future we will set
up a more streamlined way of doing this.


\section{Chopper\_simple: An ideal chopper}
\index{Optics!lens}
\mcdoccomp{optics/Chopper_simple.parms}

\texttt{Chopper\_simple} models a chopper by a blocking mathematical plane which becomes transparent'
in the specified time interval. 



\newpage
\chapter{Refractive optical components: lenses}
\label{c:lenses}

An X-ray refractive lens, often referred to as a Compund Refractive Lens (CRL), is a fairly new type of device, which has gained popularity in
the last few years. An early study showing the feasiblity of such devices may be found in~\cite{snigirev1996}. As the refractive index of X-rays
is $n\approx1$ a number of lenses, stacked together is usually necessary to bring the focal length to practical values. 
\MCX includes a few lens components, which all have slightly different charactertics.

\index{Optics|textbf}
%\section{Lens\_parab: an x-ray lens of a parabolic profile; could extend to many lenses, forming a compound refractive lens (CRL)}
\label{lens-parab}
\index{Optics!Lens\_parab}

\component{Lens\_parab}{System}{$r$,$yheight$,$xwidth$,$d$,$T$,$N$,$deltaN$,$material\_database$}
Parametrical description of an ideal x-ray lens has two uncoupled attributes $xwidth$ and $yheight$ that correspond to geometrical horizontal and vertical apertures consequently.

This is a model of an ideal x-ray lens that has a paraboloid of rotation as its profile. Such surface allows it to focus in two dimensions. A user-defined number $N$ of individual lenses configures a compound refractive lens (CRL).

The logic behind the model is the following: a photon interacts with the surface of the lens at a certain angle $\Theta$, which alters in accordance with Snell's law upon photon's entering the lens's material. The combination of the refractive process inside material (that is characterized by a material datafile) and the interaction with a geometrical surface results in a photon's new trajectory, i.e. in focusing.

$deltaN$ is a refractive index decrement $\delta$ multiplied by the number of the lenses in case of a CRL, this number allows to set in a particular value for the parameter, bypassing the material datafile (in case when it isn't available). It is also appropriate when using a thin lens approximation to a thick lens problem.  
  
\section{Lens\_parab\_Cyl: an x-ray lens of a parabolic cylinder profile; could extend to many lenses, forming a compound refractive lens (CRL)}
\label{lens-parab-cyl}
\index{Optics! Lens\_parab\_Cyl}

\component{Lens\_parab\_Cyl}{System}{$r$,$yheight$,$xwidth$,$d$,$T$,$N$,$deltaN$,$material\_database$}

This component is based on the same logical approach as the {Lens\_parab} with one significant difference - geometrical surface. Parabolic cylinder (parabolic curvature along the vertical axes only, invariant along the horizontal) allows to focus the beam in one dimension - vertical.

\section{Lens\_simple: Thin lens approximation}
\label{s:lens-simple}
\index{Optics!lens}

\mcdoccomp{optics/Lens_simple.parms}

This models a thin-lens approximation of a stack of parabolic refractive x-ray lenses

\section{Lens\_parab: Thick parabolic CRL}
\index{Optics!lens}
\mcdoccomp{optics/Lens_parab.parms}

Component model of a stack of compund refractive lenses. Each lens in the stack is modelled by two parabolic surfaces, and rays are traced through all the complete stack  taking the displace,ement of the surfaces into account. This is naturally less efficient than a thin lens approximation.

The functionality of \texttt{Lens\_parab\_Cyl}, \texttt{Lens\_parab\_rough}, and \texttt{Lens\_parab\_Cyl\_rough} will be merged into this component.

\section{Lens\_parab\_Cyl: Thick 1D-parabolic CRL}
\index{Optics!lens}
\mcdoccomp{Lens_parab_Cyl.parms}

This component and its functionality is scheduled to be merged into \texttt{Lens\_parab}


\section{Lens\_parab\_rough: Thick parabolic CRL including roughness-model}
\label{s:lens-parab-rough}
\index{Optics!lens}
\mcdoccomp{optics/Lens_parab_rough.parms}

Identical to \textbf{Lens\_parab} except it has the option of a \textit{roughness} parameter.
Roughness is simply modelled by a stochatstic, normally distributed, displacement of the normal vector of the lens surfaces.

This component and its functionality is scheduled to be merged into \textbf{Lens\_parab}

\section{Lens\_parab\_Cyl\_rough: Thick 1D-parabolic CRL including roughness-model}
\label{s:lens-parab-cyl-rough}
\index{Optics!lens}
\mcdoccomp{optics/Lens_parab_Cyl_rough.parms}

Identical to \textbf{Lens\_parab\_Cyl} except it has the option of a \textit{roughness} parameter.
Roughness is simply modelled by a stochatstic, normally distributed, displacement of the normal vector of the lens surfaces.

This component and its functionality is scheduled to be merged into \texttt{Lens\_parab}


\section{Lens\_Kinoform: refractice kinoform lens}
\index{Optics!lens}
\mcdoccomp{Lens_Kinoform.parms}

Doc. Pend.

\section{Lens\_elliptical: }
\index{Optics!lens}
\mcdoccomp{Lens_elliptical.parms}

doc. pend.



\newpage
\chapter{Reflective optical components: mirrors}
\label{c:mirrors}
\index{Optics|textbf}
This section describes advanced reflective X-ray optics
components such as mirrors.
%A description of the reflectivity of a mirror is found
%in section~\ref{ss:mirrorreflect}.
%\section{Mirrors etc.}
\label{s:mirrors}
\index{Optics|textbf}

This section describes advanced X-ray optics
components such as mirrors and analyzer crystals.
A description of the reflectivity of a mirror is found
in section~\ref{ss:mirrorreflect}.
 

\section{Mirror\_curved: Cylindrically curved mirror}
\index{Optics!mirror}

\mcdoccomp{Mirror_curved.parms}

A cylindrically curved mirror.
This component is scheduled to be merged with \texttt{Mirror_parabolic} and \texttt{Mirror_elliptic} 

\section{Mirror\_parabolic: Mirror with a parabolic curvature profile.}
\index{Optics!mirror}

\mcdoccomp{optics/Mirror_parabolic.parms}

This component is scheduled to be merged with \texttt{Mirror\_elliptic}

\section{Mirror\_elliptic: Mirror with a elliptic curvature profile.}
\index{Optics!mirror}

\mcdoccomp{optics/Mirror_elliptic.parms}

This component is scheduled to be merged with \texttt{Mirror\_parabolic}



\section{Multilayer\_elliptic: Elliptically curved mirror coated with a multilayer}
\index{Optics!multilayer, mirror}

\mcdoccomp{optics/Multilayer_elliptic.parms}

doc. pend.


\section{TwinKB\_ML: Side-by-side Kirkpatrick-Baez mirror pair}

\mcdoccomp{optics/TwinKB_ML.parms}

Models a pair of perpendicular, elliptically curved mirrors, known as a Montel-mirror or Side-by-side Kirkpatrick-Baez mirror.

