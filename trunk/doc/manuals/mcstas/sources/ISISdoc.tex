\section{ISIS\_moderator: ISIS pulsed moderators}
\label{isis-moderator}
\index{Sources!ISIS pulsed moderators}

\component{ISIS\_moderator}{S. Ansell and D. Champion, ISIS}{Face,$E0, E1,dist,xw,yh,CAngle,SAC$ }{modXsize,modYsize}{Validated. Low statistics above 20 \AA. Kink aroung 9 \AA.}

\subsection{Introduction}

The following document describes the functions obtained for models of
TS2 as described in Table~\ref{desc}:

\begin{table}[h]
\begin{center}
\begin{tabular}{|l|l|}
\hline
target & 3.4cm diameter tantalum clad tungsten \\
\hline
reflector & Be + D$_2$O (80:20) at 300K \\
\hline
Composite Moderator & H$_2$ + CH$_4$ \\
Coupled         & Groove: 3x8.\.{3} cm 26K solid-CH$_4$ \\
                & Hydrogen: 12x11cm 22K liquid H$_2$ \\
\hline
Poisoned Moderator &  solid-CH$_4$ 26K  \\
Decoupled           & Narrow: Gd poison at 2.4 cm - 8 vanes\\
                    &  Broad: 3.3 cm -- not fully decoupled \\
\hline
PreModerators & 0.85 cm and 0.75 cm H$_2$O \\
\hline
\end{tabular}
\caption{Description of Models}
\label{desc}
\end{center}
\end{table}
%Table 1: {\bf Description of Models}

TS1 model is from the tungsten target as currently installed and
positioned. The model also includes the MERLIN moderator, this
makes no significant difference to the other moderator faces.




\subsection{Using the McStas Module}

You MUST first set the environment variable `MCTABLES' to be the
full path of the directory containing the table files:
\begin{lstlisting}
BASH: export MCTABLES=/usr/local/lib/mcstas/contrib/ISIS_tables/
TCSH: setenv MCTABLES /usr/local/lib/mcstas/contrib/ISIS_tables/
\end{lstlisting}
In Windows this can be done using the `My Computer' properties and
selecting the `Advanced' tab and the Environment variables button.
This can of course be overridden by placing the appropriate moderator (h.{\it{face}}) files in the
working directory.

The module requires a set of variables
listed in Table~\ref{vars} and described below.

The {\it Face} variable determines the moderator surface that will
be viewed. There are two types of {\it Face} variable: i)  Views
from the centre of each moderator face defined by the name of the
moderator, for TS1: Water, H2, CH4, Merlin and TS2: Hydrogen,
Groove, Narrow, Broad. ii) Views seen by each beamline, for TS1:
Prisma, Maps, crisp etc. and for TS2: E1-E9 (East) and W1-W9
(West).

The \MCS distribution includes some example moderator files for TS1 (water,h2,ch4) and TS2 (broad, narrow, hydrogen, groove), but others are available at \\ \verb+http://www.isis.rl.ac.uk/Computing/Software/MC/+, including instrument specific models.

% Views seen by each beamline, for TS1 N-N (North) and S-S
% (south) and TS2: E1-E9 (East) and W1-W9 (West).


Variables {\it E0} and {\it E1} define an energy window for sampled neutrons.
This can be used to increase the statistical
accuracy of chopper and mirrored instruments. However, {\it E0} and
{\it E1} cannot be equal (although they can be close). By default these arguments
select energy in meV, if negative values are given, selection will be in terms of Angstroms.

Variables {\it dist}, {\it xw} and {\it yh} are the three
component which will determine the directional acceptance window.
They define a rectangle with centre at (0,0,dist) from the
moderator position and with width {\it xw} meters and height {\it yh} meters.
The initial direction of all the neutrons are chosen (randomly) to
originate from a point on the moderator surface and along a
vector, such that without obstruction (and gravitational effects),
they would pass through the rectangle. This should be used as a
directional guide. All the neutrons start from the surface of the
moderator and will be diverted/absorbed if they encountered other
components. The guide system can be turned off by setting {\it
dist} to zero.

The {\it CAngle} variable is used to rotate the viewed direction
of the moderator and reduces the effective solid angle of the
moderator face. Currently it is only for the horizontal plane.
This is redundant since there are beamline specific h.{\it{face}} files.

The two variables {\it modYsize} and {\it modXsize}
allow the moderators to be effectively reduced/increased. If
these variables are given  negative or zero values then they default to the actual
visible surface size of the moderators.

The last variable {\it SAC} will correct for the different solid angle seen by two
focussing windows which are at different distances from the moderator surface. The
normal measurement of flux is in neutrons/second/\AA/cm{$^2$}/str, but in a detector
it is measured in neutrons/second. Therefore if all other denominators in the flux are
multiplied out then the flux at a point-sized focus window should follow an inverse square law.
This solid angle correction is made if the {\it SAC} variable is set equal to 1, it will not be
calculated if {\it SAC} is set to zero. It is advisable to select this variable at all times as it will give the most realistic results

\subsection{Comparing TS1 and TS2}
The Flux data provided in both sets of h.{\it{face}} files is for 60 {$\mu$}Amp sources. To compare TS1 and TS2, the TS1 data must be multiplied by three (current average strength of TS1 source ~180 {$\mu$}Amps). When the 300 {$\mu$}Amp upgrade happens this factor should be revised accordingly.
\begin{table}[tbp]
\begin{center}
\begin{tabular}{|p{0.7in}|p{0.4in}|p{1.7in}|p{0.35in}|p{2.0in}|}
\hline
Variable & Type & Options & Units & Description \\
\hline Face (TS2) & char* &  i) Hydrogen Groove Narrow~Broad,
~~~~~~~~~~~~~~ii) E1-E9 W1-W9 & -- &
String which designates the name of the face \\
\hline Face (TS1) & char* &  i) H2 CH4 Merlin ~~~~~~ Water,  ii)
Maps Crisp Gem EVS HET HRPD Iris Mari Polaris Prisma Sandals Surf
SXD Tosca & -- &
String which designates the name of the face \\
\hline
E0 & float & 0$<$E0$<$E1 & meV (\AA) & Only neutrons above this energy are sampled
\\
E1 & float & E0$<$E1$<$1e10 & meV (\AA) & Only neutrons below this energy are
sampled \\
\hline
dist & float & $0< dist <\infty$ & m & Distance of focus window from
face of moderator \\
xw & float & $0<xw<\infty$ & m & x width of the focus window  \\
yh & float & $0<yh<\infty$ & m & y height of the focus window  \\
\hline
CAngle & float & -360 $< CAngle <$ 360 & $^o$ & Horizontal angle from
the normal to the moderator surface\\
\hline
modXsize & float & $0<modXsize<\infty$ & m & Horizontal size of the
moderator (defaults to actual size) \\
modYsize & float & $0<modYsize<\infty$ & m & Vertical size of the
moderator (defaults to actual size) \\
\hline
SAC & int & 0,1 & n/a & Solid Angle Correction \\
\hline
\end{tabular}
\caption{Brief Description of Variables}
\label{vars}
\end{center}
\end{table}

\subsection{Bugs}

Sometimes if a particularly long wavelength ( $> 20$ \AA) is requested there may be problems with sampling the data. In general the data used for long wavelengths should only be taken as a guide and not used for accurate simulations. At 9 \AA there is a kink in the distribution which is also to do with the MCNPX model changing. If this energy is sampled over then the results should be considered carefully.

