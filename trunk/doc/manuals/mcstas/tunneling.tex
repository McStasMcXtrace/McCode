\section{Tunneling\_sample: An incoherent inelastic scatterer}
\label{s:Tunneling_sample}
\index{Samples!Incoherent inelastic scatterer}
\index{Incoherent inelastic scattering}

\component{Tunneling\_sample}{System}{$r_{\rm i}$, $r_{\rm o}$, $h$, $r_{\rm foc}$, $x_{\rm target}$, $y_{\rm target}$, $z_{\rm target}$}{$w_x$, $h_y$, $t_z$, $w_{\rm focus}, h_{\rm focus}$, $w_{\rm foc, angle}$, $h_{\rm foc, angle}$, $\sigma_{\rm abs}$, $\sigma_{\rm inc}$, $V_0$, $f_{\rm pack}$, $f_{\rm QE}$, $f_{\rm tun}$, $\Gamma$, $E_{\rm tun}$, target\_index}{not validated}

The component {\bf Tunneling\_sample}
displays incoherent inelastic scattering as found in a number of systems, {\em e.g.}
containing mobile hydrogen. 

For the sample geometry, we default use a
hollow cylinder (which has the solid cylinder as a limiting case).
The sample dimensions are: Inner radius $r_{\rm i}$,
outer radius $r_{\rm o}$, and height $h$. This geometry is the same as 
the default for {\bf V\_sample}, see figure \ref{f:v-sample}.

As for {\bf V\_sample}, the sample geometry can be made rectangular 
by specifying the width, $w_x$, the height, $h_y$, and the thickness, $t_z$.

Also the focusing properties are the same as for {\bf V\_sample}.
For the focusing is performed as a uniform distribution on
a target sphere of radius $r_{\rm foc}$, at the position
$(x_{\rm target},y_{\rm target},z_{\rm target})$
in the local coordinate system.
The focusing can alternatively be performed on a rectangle with dimensions
$w_{\rm focus}$, $h_{\rm focus}$, or uniformly in angular space
(in a small-angle approximation),
using $w_{\rm foc, angle}$, $h_{\rm foc, angle}$.
The focusing location can be picked to be a downstream component by
specifying \verb+target_index+.

The incoherent and absorption cross sections for V are default
for the component. For other choices, the
parameters $\sigma_{\rm inc}$, $\sigma_{\rm abs}$,
and the unit cell volume $V_0$ should be specified.
For a loosely packed sample, also the packing factor, $f_{\rm pack}$
can be specified (default value of 1).

The inelastic scattering takes place as a quasielastic (Lorentzian)
component, which is chosen with probability $f_{\rm QE}$.
The broadening of the signal is given by $\Gamma$ (HWHM).
In addition, a tunneling signal is present with a probability of $f_{\rm tun}$ 
and a tunneling energy of $\pm E_{\rm tun}$. 
The tunneling peaks are weighted by the usual factor $k_{\rm f}/k_{\rm i}$.

The total scattering cross section is given by
\begin{eqnarray}
\lefteqn{\frac{d^2\sigma}{d\Omega dE_{\rm f}}(q,\omega) = \frac{\sigma_{\rm inc}}{4\pi}
\times \left\{ (1-f_{\rm QE}-f_{\rm inel}) \delta(\hbar \omega) \right. }  \\
 &+& f_{\rm QE} \frac{\Gamma}{(\hbar\omega)^2+\Gamma^2}
 + \left.\frac{f_{\rm inel}}{2} \frac{k_{\rm f}}{f_{\rm i}} 
   \left[\delta(\hbar\omega-E_{\rm tun}) + \delta(\hbar\omega+E_{\rm tun}) \right] \right\} \nonumber
\end{eqnarray}

The component takes care that 
$f_{\rm QE} + f_{\rm tun} \leq 1$, otherwise an error is returned.

The component accounts for absorption, 
but not multiple scattering. To obtain
intensities similar to real measured ones, we therefore do not 
take attenuation from scattering into account for the outgoing
neutron ray.

