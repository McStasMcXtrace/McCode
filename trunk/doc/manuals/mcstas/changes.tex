% Emacs settings: -*-mode: latex; TeX-master: "manual.tex"; -*-

\chapter{New features in \MCS \version\ }
\label{c:changes}

%TODO FIXME UPDATE!!!!

This version of \MCS implements both new features, as well as many bug corrections. Bugs are reported and traced using the \MCS Trac Ticket system \cite{mczilla_webpage}. We will not present here an extensive list of corrections, and we let the reader refer to this bug reporting service for details. Only important changes are indicated here.

Of course, we can not guarantee that the software is bullet proof, but we do our best to correct bugs, when they are reported.\index{Bugs}

The listing below contains the \MCS \version CHANGES document.

\begin{lstlisting}
Changes in McStas v2.1, September, 2014

 McStas 2.1 is the second release in the 2.x series and fixes various issues with McStas 2.0,
 plus provide important new developments.

 IMPORTANT: Please ensure to use our "migration scripts" for McStas 2.0 and 1.12c if you want
 these to co-exist with 2.1. Also, please read the platform-specific notes below!

 Thanks:
 - Thanks to all contributors of components, instruments etc.! This is what Open Source
   and McStas is all about!
 - Thanks to Jonas Stein from Uni Cologne for helping us modernize the TeX documentation!
 
 General:
 - Via NeXus libraries and a new --format=Mantid setting for the mcdisplay tool, we are now
   able to generate NeXus/HDF files that can be read and treated by Mantid! Mcdisplay generates
   a Mantid IDF description for embedding in the NeXus file, including the usual McStas instrument
   geometry. Please read the specific guide for this method, documented in the McStas manual, Chapter 6.
 - Use of STATE_PARAMETERS and POLARIZATION parameters is no longer supported!
   If you have home-grown components using these, you will get this type of error with 
   McStas 2.1:

      Translating instrument definition 'BNL_H8.instr' into C ...
      mcstas -t -o BNL_H8.c BNL_H8.instr
      Compiling simulation BNL_H8.instr
      Arm.comp is using STATE PARAMETERS
      	       mcstas-2.1 does NOT support this keyword. Please remove line 36
      mcstas-2.1: 1 Errors encountered during parse of BNL_H8.instr.
      ** Error exit **

      
   As the error message suggests, this can be fixed by simply removing/commenting out
   the STATE and POLARISATION parameter lines in the top of the component, i.e.:
   /* STATE PARAMETERS (x,y,z,vx,vy,vz,t,s1,s2,p) */
   /* POLARISATION PARAMETERS (sx,sy,sz) */
   
 - In case of "negative time" propagation we no longer ABSORB these neutrons but rather restore
   the incoming neutron state, leaving it untouched for the following components. This avoids
   various types of shadowing effects, e.g. when placing parallel monitors that are not in a 
   GROUP. This is furhter in preparation for the forthcoming ASSEMBLY keyword which will work
   roughly like an "iterative group" that loops on the ASSEMBLY of components until no furhter
   scattering is achieved, but always scatters first on the component which is closest (in time)
   to the neutron.

 - "Scatter logging": With this release we provide this special ensemble of instrument examples
   - Test_Scatter_log_losses.instr
   - Test_Scatter_log_mvalues.instr
   - Test_Scatter_log_ssw_mcnp.instr
   that together with the (misc cathegory) components
   - Scatter_log_iterator.comp
   - Scatter_log_iterator_stop.comp
   - Scatter_logger.comp
   - Scatter_logger_stop.comp
   provide a powerful post-processing mechanism, mainly for guide systems - that for instance
   can be used to study the non-propagated, i.e. absorbed neutron flux along the guide. For
   more information, a paper by Esben Klinkby et. al. will be available in one of the next 
   issues of Journal of Physics: Conference Series, proceedings of the NOP&D 2013 meeting in
   Ismaning. Slides from Esbens talk are available here:
   http://orbit.dtu.dk/files/57025387/prod11375088187360.NO_P_v8.pdf

 - As of this release, McStas is distributed using easy-to-use meta-packages for the Windows
   and Mac OS X systems. (these already existed for the deb and rpm based Linux packages, see
   below) The content of these packages are:

   * Windows:
     - See http://www.mcstas.org/download/install_windows for details
     - Strawberry Perl (including a gcc compiler)
     - Perl PPD extensions for PDL, PGPLOT and Tk
     - The core mcstas package
     - The mcstas component package, with comps, instruments, datafiles
     - The mcstas Perl tools package with mcgui, mcplot etc.
     - The mcstas manuals package with PDF docs
     - For VRML and X3D viewing of instruments, we recommend Instant Player from
       http://www.instantreality.org
     - ! The mcstas Python tool packages are installed by default, but if you want to 
       make use of our OPTIONAL Python based packages, their dependencies need to be installed
       by hand !
       - Needed dependencies for the Python packages are all available in the Python(x,y)
       	 package available from https://code.google.com/p/pythonxy/

   - ! We are only distributing a 32bit package for windows, since we feel the 64-bin gcc on
       windows is not mature - gives problems in a few, specific cases. In any case, the 32-bit
       McStas for windows works perfectly fine on a 64-bit operating system !  

   - ! If you experience strange behavior from perl/mcgui on Windows like
       	    a) ppm shell not being able to install the extra Perl packages
            b) mcgui not being able to run / compile a simulation (all you get is mcrun -- help output)
     	    c) mcgui not being able to access local component files
	  - Solution can be to ensure your user has "full control" to the executables in
	    c:\strawberry\perl\bin (Right-click, Properties, Security, Authenticated users -> Edit)

   * Mac OS X:
     - IMPORTANT: Please ensure to use our "migration scripts" for McStas 2.0 and 1.12c if you want
       these to co-exist with 2.1. They MUST be applied before installing 2.1!
     - See http://www.mcstas.org/download/install_mac_os_x for details
     - A McStas 'app' bundle in /Applications/McStas-2.1.app - you can uninstall all of McStas 2.1
       by dragging this bundle to the trash bin.
     - Tk perl package for mcgui
     - SciPDL package for mcplot, including the core pgplot package
     - The core mcstas package
     - The mcstas component package, with comps, instruments, datafiles
     - The mcstas Perl tools package with mcgui, mcplot etc.
     - The mcstas manuals package with PDF docs
     - For VRML and X3D viewing of instruments, we recommend Instant Player from
       http://www.instantreality.org
     - ! If you want to use our OPTIONAL Python based packages, select these for installation
       in the Metapackage installer !
       - For the Python mcrun, you need to install PyYAML from http://www.pyyaml.org/wiki/PyYAML
       - Most other needed dependencies for the Python packages are available through the 
       	 SciPy superpack available from http://fonnesbeck.github.io/ScipySuperpack/
       - You may also choose the commerical Anaconda or Enthought Python distributions
       
   - ! The install location of the McStas components and libraries on Mac OS X has changed! Your files
       are now placed in /Applications/McStas-2.1.app/Contents/Resources/mcstas/2.1/ - but for ease
       of use, a link is also available in /usr/local/mcstas/2.1/ (Note that we have dropped 'lib' in 
       that path.) Links to binaries are in /usr/local/bin.

   * For Linux (Debian and RPM based)
     - IMPORTANT: Please ensure to use our "migration scripts" for McStas 2.0 and 1.12c if you want
       these to co-exist with 2.1. They MUST be applied before installing 2.1! 
     - See http://www.mcstas.org/download/install_linux for details
     - For some RPM based systems, you may futher need to install the EPEL extensions, see https://fedoraproject.org/wiki/EPEL
     - For Deb and RPM type systems we provide a set of APT and YUM repositories. Simply follow the
       instructions on the link provided to add the relevant repository to your Linux.
     - Afterwards, simply choose to install mcstas-suite-perl (perl tools), mcstas-suite-python 
       (python tools) or mcstas-suite (both toolsets) and all dependencies should be taken care of.

   - ! The install location of the McStas components and libraries on Linux X has changed! 
       * On Debian, your files are now placed in /usr/share/mcstas/2.1/ - links to binaries are in /usr/bin
       * On RedHat, your files are now placed in /usr/local/mcstas/2.1/ - links to binaries are in /usr/local/bin
       (Note that we have dropped 'lib' in the mcstas system path.)


  
 - A given McStas 2.x release is "self-contained" and can be installed in parallel with other versions
   of McStas and McXtrace 
 - We provide a "migration tool" for Unix based systems (i.e. Linux & Mac) to let your already
   installed McStas 1.x or 2.0 co-exist with the new 2.1
 - Using McStas with NeXus does no longer require recompilation if you only want to _write_ NeXus.
   (Simply install NeXus and update your MCSTAS_CFLAGS to include -lNeXus -DUSE_NEXUS). For the
   conversion of files using mcformat NeXus must still be available at compile time

 Core & runtime code:
 - Backward compatibility 1: As written above, STATE and POLARIZATION parameters are no longer supported!
 - Complete rework of the DETECTOR_OUT infrastructure:
   * Simplified code-structure
   * Proper support for MPI when generating NeXus event lists
 - New logic wrt. negative time propagation: Neutrons are restored rather than ABSORB'ed
 - Improvement of the interpolation routines e.g. for magnetic fields: search is performed roughly in constant time
 - Read_Table: better search for files, including from full executable path, from source.instr location, from 
   PWD and from McCode lib.
 - ref-lib.c: If components use alpha=0 and w=0 in parametrizing the reflectivity, the StdReflecFunc
   applies a phenomenological fit of Swiss Neutronics mirrors (above m=3 including a 2nd order beta term). The fits
   were done by Henrik Jacobsen, University of Copenhagen.

 Tools:
 - Since 2.0 mcgui Perl and mcrun Python has been generating time-stamped output directories if no output dir 
   was specified. The legacy mcrun Perl now exhibit this behavior also. This is balanced by the possiblitiy
   to choose . (dot) as the output directory - then the legacy behaviour of storing data in the working
   directory is enabled.

 Documentation:
 - Modernized and facelifted TeX documentation in the manuals, which include slightly updated Component manual.
   A few components-docs parameter tables are now extracted from mcdoc and included in the manual. 

 New Components:
 - Rotator and Derotator - components that allow a section of the instrument to rotate in time.
 - New ESS_moderator component by P Willendrup, E Klinkby and T Schoenfeldt DTU
   * new ess_source-lib with brilliance defintitions corresponding to Mezei 2001, TDR and 2014 brilliances
   * support for geometrical brilliance variation (i.e. "pancake" geometry)
   * much simplified input-parameter setup
   * various fixes wrt. McStas 2.0 and later updated source modules
 - Monitor_Sqw - generates an Sqw between a sample and a detector, a kind of inelastic resolution monitor.
   (Emmanuel Farhi)
 - Analyzer_ideal.comp, ideal analyzer for Spin-Echo techniques (Erik Knudsen)
 - Pol_pi_2_rotator.comp, idealized pi/2 flipper for Spin-Echo (Erik Knudsen)
 - Pol_triafield.comp, idealized triangular coil (Delft) for Spin-Echo (Erik Knudsen)
 - Scatter_log_iterator.comp, see above notice about "Scatter logging"
 - Scatter_log_iterator_stop.comp, see above notice about "Scatter logging"
 - Scatter_logger.comp, see above notice about "Scatter logging"
 - Scatter_logger_stop.comp, see above notice about "Scatter logging"
 - SANS_benchmark2.comp, benchmarked SANS component from Heinrich Frielinghaus, FZJ

 Updated Components:
 - Single_crystal and PowderN have both been optimized for speed - especially in connection with using
   the SPLIT keyword:
     * The first precomputations relating to structure and crossections is kept for each incoming netron,
       so that the component may skip these calculations for the next SPLIT version of the neutron (there
       is a built-in check for whether that kind of neutron was processed before). This means that a set'
       of SPLIT neutron may be scattered across ALL reflection that are allowed for that specific incoming
       direction and wavelength. Further, the components estimate the optimal value of the SPLIT. For large
       single crystal unit cells (i.e. protein crystals etc.), speedups of a factor of 500 have been observed!
       The method works with all orders of scattering, even though the largest gain is seen with order=1.
     * Use is easy, e.g.
          SPLIT 43 COMPONENT SX=Single_crystal(order=1, ...)
 - Single_crystal now allows to be used as a Powder, mainly for benchmarking purposes - at the point of scattering,
   the "crystallite" is randomly oriented. Final validation is pending, but in this setting, the component already
   gives the same order of manitude scattering as PowderN, where this powder ring "intergration" has been 
   implemented using a completely different strategy - provided the right values of mosaicity and delta_d_d are
   given!
 - ISIS moderator component allow specification of the accellerator current. The component performance
   defaults to the actual, relevant accellerator setting for each target station (TS). All TS1 and TS2
   moderator input files are now included with McStas and have meaningful names with reference to the 
   relevant TS. 
 - Source_gen now contains LLB and FRM-II source parameters in the component help.
 - Monitor_nD contains infrastructure for handling events in Mantid NeXus files.
 - Isotropic_Sqw - various fixes, including optimized memory usage
 - Lens component - support for abstract geometries via the OFF/PLY input format
 - Guide_anyshape - various improvements
 - FermiChopper - fixed bug in tangent intersection
 - Elliptic_guide_gravity - support for curvature via horizontal, vectorial addition to gravity

 Instruments:
 - Various instruments have been included, allowing to estimate pulsed-source average and peak brilliance.
 - ISIS_SANS2d instrument contributed by R. Heenan, ISIS
 - Test_Guide_Curved - test instrument for simple, curved guide types
 - HZB_FLEX contributed by Klaus Habicht and Markos Skoulatos, HZB
 - SEMSANS_instrument.instr by Morten Sales, HZB
 - Test_Scatter_log_losses.instr, see above notice about "Scatter logging"
 - Test_Scatter_log_mvalues.instr, see above notice about "Scatter logging"
 - Test_Scatter_log_ssw_mcnp.instr, see above notice about "Scatter logging"
 - FZJ_BenchmarkSfin2.instr by Heinrich Frielinghaus, FZJ. Test instrument for the SANS_benchmark2.comp
 - templateNMX.instr, templateLaue variant demonstrating the Single_crystal speed improvement mentioned above

\end{lstlisting}
