\chapter{Links to other computing codes}

This chapter provides information on interoperability of \MCS with
other codes, and provide links to other sources of information on this topic.

\section{\MCS and MANTID}
As of \MCS 2.1 and MANTID 3.2, the teams of the two codes provide a mechanism to transfer simulated \MCS
events and monitor data to MANTID, via NeXus files.

\subsection{System requirements}
To enable the software link, you need the following codes installed on
your system:
\begin{itemize}
\item{\MCS 2.1 or newer}
\item{NeXus libraries from \verb+http://nexusformat.org+ \footnote{For Mac OS X you may get
    better milage via the Mantid github site}}
\item{Mantid 3.2 or newer}
\item{On Mac OS X 10.8 and newer you may need to install one of the
    below compilers, as the default clang on OS X causes problems for
    the link, especially when used together with MPI}
  \begin{itemize}
  \item{Intel C}
  \item{gcc from \verb+http://hpc.sourceforge.net+}
  \end{itemize}
\end{itemize}

\subsection{Requirements for the instrument file}
A special naming convention in the instrument file is needed for the
automatic transfer of geometry to a Mantid IDF file:

\begin{itemize}
\item The location of the source must be indicated by a component
  named sourceMantid. Note that in the case of a curved instrument
  geometry, you should probably add an Arm where Mantid 'expects' the
  source to be, i.e.: defining a location displaced the correct
  source-sample distance, parallel to the incoming beam direction at
  the sample.
\item The location of the sample must be indicated by a component
  named sampleMantid
\item One or more Monitor\_nD components need to be added in either
  rectangular- or cylindrical geometry and with a set of special
  flags, as shown below
\item Rectangular monitor
  \begin{mcstas}
    COMPONENT nD_Mantid_0 = Monitor_nD(
    options ="mantid square x limits=[-0.2 0.2] bins=128 y limits=[-0.2 0.2] bins=128, neutron pixel t, list all neutrons",
    xmin = -0.2,
    xmax = 0.2,
    ymin = -0.2,
    ymax = 0.2,
    restore_neutron = 1,
    filename = "bank01_events.dat")
  AT (0, 0, 3.2) RELATIVE sampleMantid 
\end{mcstas}
\item Cylindrical monitor
  \begin{mcstas}
    COMPONENT nD_Mantid_01 = Monitor_nD(xwidth=(4.0-0.0005-0.00002)*2, yheight=3,
    options="mantid banana, theta limits=[-73.36735 73.36765] bins=100, y limits=[-1.5 1.5] bins=300, neutron pixel t, list all neutrons", restore_neutron=1)
    AT (0,0,0) RELATIVE center_det
\end{mcstas}
\end{itemize}

Other, ordinary \MCS monitors will also be visible in the resulting
Mantid workspace, but will not be easily processible using the Mantid
TOF data reduction schemes.

\subsection{Compiling and running your simulation for Mantid output}
Geometry information in Mantid is handled via a so-called Instrument
Definition File (IDF) in xml-format, possibly embedded in a NeXus
file. The creation of the IDF and related NeXus file is handled in the
following steps:
\begin{enumerate}
\item First of all, enable NeXus in the compilation process:
 \begin{mcstas} 
   export MCSTAS_CFLAGS="-g -lm -O2 -DUSE_NEXUS -lNeXus"
 \end{mcstas}
\item Compile the instrument via the mcrun utility (add \verb+--mpi+
  if you need parallelization support):
 \begin{mcstas} 
   mcrun -c ILL_H16_IN5_Mantid.instr -n0
 \end{mcstas}
\item Generate the IDF via the mcdisplay utility (and press enter
  until the simulation runs):
 \begin{mcstas} 
   mcdisplay --format=Mantid ILL_H16_IN5_Mantid.instr -n0
 \end{mcstas}
\item Finally, run a simulation with NeXus output:
 \begin{mcstas} 
   mcrun --format=NeXus ILL_H16_IN5_Mantid.instr 
 \end{mcstas} 
\end{enumerate}

\subsection{Looking at instrument output in Mantid}




\section{\MCS and MCNP(X)}
