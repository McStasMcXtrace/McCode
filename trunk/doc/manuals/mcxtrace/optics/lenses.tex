\section{Lens\_parab: an x-ray lens of a parabolic profile; could extend to many lenses, forming a compound refractive lens (CRL)}
\label{lens-parab}
\index{Optics!Lens\_parab}

\component{Lens\_parab}{System}{$r$,$yheight$,$xwidth$,$d$,$T$,$N$,$deltaN$,$material\_database$}
Parametrical description of an ideal x-ray lens has two uncoupled attributes $xwidth$ and $yheight$ that correspond to geometrical horizontal and vertical apertures consequently.

This is a model of an ideal x-ray lens that has a paraboloid of rotation as its profile. Such surface allows it to focus in two dimensions. A user-defined number $N$ of individual lenses configures a compound refractive lens (CRL).

The logic behind the model is the following: a photon interacts with the surface of the lens at a certain angle $\Theta$, which alters in accordance with Snell's law upon photon's entering the lens's material. The combination of the refractive process inside material (that is characterized by a material datafile) and the interaction with a geometrical surface results in a photon's new trajectory, i.e. in focusing.

$deltaN$ is a refractive index decrement $\delta$ multiplied by the number of the lenses in case of a CRL, this number allows to set in a particular value for the parameter, bypassing the material datafile (in case when it isn't available). It is also appropriate when using a thin lens approximation to a thick lens problem.  
  
\section{Lens\_parab\_Cyl: an x-ray lens of a parabolic cylinder profile; could extend to many lenses, forming a compound refractive lens (CRL)}
\label{lens-parab-cyl}
\index{Optics! Lens\_parab\_Cyl}

\component{Lens\_parab\_Cyl}{System}{$r$,$yheight$,$xwidth$,$d$,$T$,$N$,$deltaN$,$material\_database$}

This component is based on the same logical approach as the {Lens\_parab} with one significant difference - geometrical surface. Parabolic cylinder (parabolic curvature along the vertical axes only, invariant along the horizontal) allows to focus the beam in one dimension - vertical.
