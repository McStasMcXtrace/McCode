\section{Filter: A general absoprtion filter model}
\label{filter}
\index{Optics!Filter}

\component{Filter}{System}{$xwidth$, $yheight$, $zdepth$ $material_datafile$}{$options$}{not validated, absorption filter}

This component is a filter in the shape of a ectangular block. Given an input file containing material parameters.
Neccessary paramters are nominal density and a parametrization of mu as a function of wavelength (or energy).

The model is very simple: Firstly the X-ray is traced to find intersection points between ray and filter (0 or 2).
If no intersection is found the xray is left untouched and nothing further happens.
Assuming the ray intersects the filter: Secondly, the path length d$l$ within the filter is computed.
Thirdly a $\mu = f(\lambda,\mathrm{material})$ is computed by interpolating in a datafile, and the xray weight is adjusted according to $p=p\exp(-\mathrm{d}l*\mu)$. The xray is left at the point where it exits the filter block (the $2$nd ntersection).

Example data files corresponding to all elements up to $Z=92$ are distributed with \MCX in the
\verb+MCXTRACE/data+ directory as \verb+*.txt+ files. These tables have been extracted from the NIST x-ray database.
To generate other datafiles see below-
from the same source a simple shell script: \verb+MCSTAS/data/get_xray_db_data+ is also distributed with \MCX
Running this script will connect to the NIST webiste and download a
\verb+.html+ file. This output must now be modified such that \verb+html+-tags
are removed and all header lines begin with $\#$

\subsection{Example}
\label{getNISTdata}
This is an example of how to download and generate datafiles for the \verb+Filter.comp+ and others.

The distributed tables have been extracted from the NIST x-ray database. To ease generation of more dtafiles
from the same source a simple shell script: \verb+MCSTAS/data/get_xray_db_data+ is also distributed with \MCX

Running this script will connect to the NIST webiste and download a \verb+.html+ file. This output must now be modified wuch that \verb+html+-tags
are removed and all header lines begin with $\#$.

\begin {verbatim}
 /usr/local/lib/mcxtrace/data/get_xray_db_data 3 ouput.html
\end{verbatim}
where the second parameter (3) is the atom number of the material, for which we want to generate a datafile.
Now open the generated datafile (output.dat) with your favourite text editor and make sure the file ends up looking like this
\begin{verbatim}
#Li (Z 3)
#Atomic weight: A[r]  6.941000
#Nominal density: rho 5.3300E-01
#    σ[a](barns/atom) = [μ/ρ](cm^2 g^-1)  ×  1.15258E+01
#    E(eV) [μ/ρ](cm^2 g^-1) = f[2](e atom^-1)  ×  6.06257E+06
#    2 edges. Edge energies (keV):
#
#
#    K      5.47500E-02  L I    5.34000E-03
#
#Relativistic correction estimate f[rel] (H82,3/5CL) = -9.8613E-04,
#    -6.0000E-04 e atom^-1
#    Nuclear Thomson correction f[NT] = -7.1131E-04 e atom^-1
#
#━━━━━━━━━━━━━━━━━━━━━━━━━━━━━━━━━━━━━━━━━━━━━━━━━━━━━━━━━━━━━━━━━━━━━━━━━━━━━━━
#Form Factors, Attenuation and Scattering Cross-sections
#Z=3, E = 0.001 - 433 keV
#
#      E            f[1]          f[2]        [mu/rho]      [sigma/rho]      [mu/rho]      [mu/rho][K]      lambda
#                                      Photoelectric Coh+inc      Total
#     keV        e atom^-1      e atom^-1   cm^2 g^-1       cm^2 g^-1      cm^2 g^-1   cm^2 g^-1     nm
5.233200E-03  9.08733E-01  0.0000E+00  0.0000E+00  2.3914E-07  2.3914E-07  0.000E+00  2.369E+02
5.313300E-03  8.59283E-01  0.0000E+00  0.0000E+00  2.5404E-07  2.5404E-07  0.000E+00  2.333E+02
5.334660E-03  8.03599E-01  0.0000E+00  0.0000E+00  2.5813E-07  2.5813E-07  0.000E+00  2.324E+02
5.366700E-03  8.56971E-01  1.0769E-01  1.2165E+05  2.6435E-07  1.2165E+05  0.000E+00  2.310E+02
.
.
.
3.788588E+02  3.00000E+00  3.9121E-08  6.2602E-07  8.4389E-02  8.4390E-02  6.123E-07  3.273E-03
4.050001E+02  3.00000E+00  3.3438E-08  5.0054E-07  8.2127E-02  8.2128E-02  4.895E-07  3.061E-03
4.329451E+02  3.00000E+00  2.8581E-08  4.0022E-07  7.9892E-02  7.9892E-02  3.913E-07  2.864E-03
\end{verbatim}
Please make sure you don't forget to remove the html-tags in the bottom of the file as well. In the future we will set
up a more streamlined way of doing this.

\begin{table}
  \begin{center}
  {\let\my=\\
    \begin{tabular}{|l|p{0.7\textwidth}|}
    \hline
    File name & Description \\
    \hline
    Be.txt & Beryllium filter block\\
    Si.txt & Silica filter block\\
    Al.txt & Aluminium.txt\\
    \hline
    \end{tabular}
    \caption{Some material data file to be used with the Filter component}
    \label{t:source-params}
  }
  \end{center}
\end{table}

