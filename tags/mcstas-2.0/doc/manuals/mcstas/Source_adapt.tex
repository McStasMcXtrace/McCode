\section{Source\_adapt: A neutron source with adaptive importance sampling}
\label{s:Source_adapt}
\label{s:source-adapt}
\index{Optimization}
\index{Sources!Adaptive source}

\component{Source\_adapt}{K. Nielsen}{$x_{min}$, $x_{max}$, $y_{min}$, $y_{max}$, $E0$, $dE$, dist, $xw$, $yh$, $\Phi$}{$\alpha$, $\beta$ (plenty, default values are ok)}{partially validated}

{\bf Source\_adapt} is a neutron source that uses adaptive
importance sampling to improve the efficiency of the simulations. It
works by changing on-the-fly the probability distributions from which
the initial neutron state is sampled so that samples in regions that
contribute much to the accuracy of the overall result are preferred over
samples that contribute little. The method can achieve improvements of a
factor of ten or sometimes several hundred in simulations where only a
small part of the initial phase space contains useful neutrons.
This component uses the correlation between neutron energy,
initial direction and initial position.

The physical characteristics of the source are similar to those of
{\bf Source\_simple} (see section~\ref{source-simple}). The source is a thin
rectangle in the $x$-$y$ plane with a flat energy spectrum in a
user-specified range. The flux, $\Phi$, per area per steradian per
{\AA}ngstr{\o}m per second is specified by the user.

The initial neutron weight is given by Eq. (\ref{proprule}) using
$\Delta\lambda$ as the total wavelength range of the source.
A later version of this component will probably include a
$\lambda$-dependence of the flux.

We use the input parameters \textit{dist}, \textit{xw}, and \textit{yh}
to set the focusing as for Source\_simple (section~\ref{source-simple}).
The energy range will be from $E_0 - dE$ to $E_0 + dE$.
\textit{filename} is used to give the name of a file in which to
output the final sampling destribution, see below.
$N_{\rm eng}$, $N_{\rm pos}$, and $N_{\rm div}$
are used to set the number of bins in each dimensions.
Good general-purpose values for the optimization parameters are
$\alpha = \beta = 0.25$. The number of bins to choose will depend on the
application. More bins will allow better adaption of the sampling, but
will require more neutron histories to be simulated before a good
adaption is obtained. The output of the sampling distribution is only
meant for debugging, and the units on the axis are not necessarily
meaningful. Setting the filename to \verb+NULL+ disables the output of
the sampling distribution.

\subsection{Optimization disclaimer}

A warning is in place here regarding potentially wrong results
using optimization techniques.
It is highly recommended in any case to benchmark 'optimized' simulations
against non-optimized ones, checking that obtained results are the same,
but hopefully with a much improved statistics.

\subsection{The adaption algorithm}

The adaptive importance sampling works by subdividing the initial
neutron phase space into a number of equal-sized bins. The division is
done on the three dimensions of energy, horizontal position, and
horizontal divergence, using $N_{\rm eng}$, $N_{\rm pos}$, and $N_{\rm
  div}$ number of bins in each dimension, respectively. The total number
of bins is therefore
\begin{equation}
N_{\rm bin} = N_{\rm eng} N_{\rm pos} N_{\rm div}
\end{equation}
Each bin $i$ is assigned a sampling weight $w_i$; the probability of
emitting a neutron within bin $i$ is
\begin{equation}
P(i) = \frac{w_i}{\sum_{j=1}^{N_{\rm bin}} w_j}
\end{equation}
In order to avoid false learning, the sampling weight of a bin is
kept larger than $w_{\rm min}$, defined as
\begin{equation}
w_{\rm min} = \frac{\beta}{N_{\rm bin}}\sum_{j=1}^{N_{\rm bin}}w_j,\qquad
    0 \leq \beta \leq 1
\end{equation}
This way a (small) fraction $\beta$ of the neutrons are sampled
uniformly from all bins, while the fraction $(1 - \beta)$ are sampled in an adaptive way.

Compared to a uniform sampling of the phase space (where the probability
of each bin is $1/N_{\rm bin}$), the neutron weight
must be adjusted as given by (\ref{probrule})
\begin{equation}
\pi_1 = \frac{P_1}{f_{\rm MC,1}} =\frac{1/N_{\rm bin}}{P(i)} =
    \frac{\sum_{j=1}^{N_{\rm bin}} w_j}{N_{\rm bin} w_i} ,
\end{equation}
where $P_1$ is understood by the "natural" uniform sampling.

In order to set the criteria for adaption, the {\bf Adapt\_check} component is
used (see section~\ref{s:adapt_check}). The source attemps to sample
only from bins from which neutrons are not absorbed prior to the
position in the instrument at which {\bf Adapt\_check} is
placed. Among those bins, the algorithm attemps to minimize the variance
of the neutron weights at the {\bf Adapt\_check} position. Thus bins that
would give high weights at the {\bf Adapt\_check} position are sampled more
often (lowering the weights), while those with low weights are sampled
less often.

Let $\pi = p_{\rm ac}/p_0$ denote the ratio between the neutron weight $p_1$ at
the {\bf Adapt\_check} position and the initial weight $p_0$ just after the
source. For each bin, the component keeps track of the sum $\Sigma$ of
$\pi$'s as well as of the total number of neutrons $n_i$ from that
bin. The average weight at the {\bf Adapt\_source} position of bin $i$ is thus
$\Sigma_i/n_i$.

We now distribute a total sampling weight of $\beta$ uniformly
among all the bins, and a total weight of $(1 - \beta)$ among bins in
proportion to their average weight $\Sigma_i/n_i$ at the {\bf Adapt\_source}
position:
\begin{equation}
w_i = \frac{\beta}{N_{\rm bin}} +
    (1-\beta) \frac{\Sigma_i/n_i}{\sum_{j=1}^{N_{\rm bins}} \Sigma_j/n_j}
\end{equation}
After each neutron event originating from bin $i$, the sampling weight $w_i$
is updated.

This basic idea can be improved with a small modification. The problem
is that until the source has had the time to learn the right sampling
weights, neutrons may be emitted with high neutron weights (but low
probability). These low probability neutrons may account for a large fraction of
the total intensity in detectors, causing large variances in the
result. To avoid this, the component emits early neutrons with a lower
weight, and later neutrons with a higher weight to compensate. This way
the neutrons that are emitted with the best adaption contribute the most
to the result.

The factor with which the neutron weights are adjusted is given by a
logistic curve
\begin{equation}
  F(j) = C\frac{y_0}{y_0 + (1 - y_0) e^{-r_0 j}}
\end{equation}
where $j$ is the index of the particular neutron history, $1 \leq j
\leq N_{\rm hist}$. The constants $y_0$, $r_0$, and $C$ are given by
\begin{eqnarray}
  y_0 &=& \frac{2}{N_{\rm bin}} \\
  r_0 &=& \frac{1}{\alpha}\frac{1}{N_{\rm hist}}
     \log\left(\frac{1 - y_0}{y_0}\right) \\
  C &=& 1 + \log\left(y_0 + \frac{1 - y_0}{N_{\rm hist}}
     e^{-r_0 N_{\rm hist}}\right)
\end{eqnarray}
The number $\alpha$ is given by the user and specifies (as a fraction
between zero and one) the point at which the adaption is considered
good. The initial fraction $\alpha$ of neutron histories are emitted
with low weight; the rest are emitted with high weight:
\begin{equation}
  p_0(j) =
    \frac{\Phi}{N_{\rm sim}} A \Omega \Delta\lambda
    \frac{\sum_{j=1}^{N_{\rm bin}} w_j}{N_{\rm bin} w_i}
    F(j)
\end{equation}
The choice of the constants $y_0$, $r_0$, and $C$ ensure that
\begin{equation}
\int_{t=0}^{N_{\rm hist}} F(j) = 1
\end{equation}
so that the total intensity over the whole simulation will be correct

Similarly, the adjustment of sampling weights is modified so that the
actual formula used is
\begin{equation}
w_i(j) = \frac{\beta}{N_{\rm bin}} +
    (1-\beta) \frac{y_0}{y_0 + (1 - y_0) e^{-r_0 j}}
     \frac{\psi_i/n_i}{\sum_{j=1}^{N_{\rm bins}} \psi_j/n_j}
\end{equation}

\subsection{The implementation}

The heart of the algorithm is a discrete distribution $p$. The
distribution has $N$ \emph{bins}, $1\ldots N$. Each bin has a value
$v_i$; the probability of bin $i$ is then $v_i/(\sum_{j=1}^N v_j)$.

Two basic operations are possible on the distribution. An \emph{update}
adds a number $a$ to a bin, setting $v_i^{\rm new} = v_i^{\rm old} +
a$. A \emph{search} finds, for given input $b$, the minimum $i$ such
that
\begin{equation}
 b \leq \sum_{j=1}^{i} v_j.
\end{equation}
The search operation is used to sample from the distribution p. If $r$
is a uniformly distributed random number on the interval
$[0;\sum_{j=1}^N v_j]$ then $i = {\rm search}(r)$ is a random number
distributed according to $p$. This is seen from the inequality
\begin{equation}
\sum_{j=1}^{i-1} v_j < r \leq \sum_{j=1}^{i} v_j,
\end{equation}
from which $r \in [\sum_{j=1}^{i-1} v_j; v_i + \sum_{j=1}^{i-1} v_j]$
which is an interval of length $v_i$. Hence the probability of $i$ is
$v_i/(\sum_{j=1}^N v_j)$.
The update operation is used to
adapt the distribution to the problem at hand during a simulation. Both
the update and the add operation can be performed very efficiently.

As an alternative, you may use the {\bf Source\_Optimizer} component
(see section \ref{source-optimizer}).
