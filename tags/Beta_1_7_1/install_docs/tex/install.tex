\documentclass[a4paper,12pt]{article}
\usepackage{html}
\usepackage[dvips]{graphicx}
\title{McStas installation instructions}
\author{Peter
  Willendrup,\\\htmladdnormallink{peter.willendrup@risoe.dk}{mailto:peter.willendrup@risoe.dk}}
\begin{document}
\maketitle
\section{Introduction}
This document describes installation of the
\htmladdnormallink{McStas}{http://mcstas.risoe.dk} package, including
some information on installing other required pieces of software.\\\ \\The \htmladdnormallink{McStas}{http://mcstas.risoe.dk} package is
available in three different distribution packages, e.g.
\begin{itemize}
\item{\texttt{mcstas-1.7-src.tar.gz}\\Source code package for
    building \htmladdnormallink{McStas}{http://mcstas.risoe.dk} on
  (at least) Linux and Windows 2000. - Refer to section \ref{src}}
\item{\texttt{mcstas-1.7-i686-unknown-Linux.tar.gz}\\Binary package
  for Linux systems, currently built on Debian GNU/Linux 3.0 'woody'. 
  Should work on most Linux setups.
 - Refer to section \ref{linbin}}
\item{\texttt{mcstas-1.7-i686-unknown-Win32.zip}\\Binary package
  for Win32 systems, currently built on Microsoft Windows 2000
  professional, using the \texttt{gcc} 2.95 compiler from 
  \htmladdnormallink{Bloodshed Dev-C++ 5 Beta
    7}{http://www.bloodshed.net/dev/devcpp.html}
   - Refer to section \ref{winbin}}
\end{itemize}
\section{Source code build}
\label{src}
The \htmladdnormallink{McStas}{http://mcstas.risoe.dk} package is
beeing co-developed for mainly Linux and Windows systems, however 
the Linux build instructions below will work on most Unix systems.
\subsection{Windows build}
\begin{itemize}
\item{Start by unpacking the \texttt{mcstas-1.7-src.tar.gz} package using
e.g. \htmladdnormallink{Winzip}{http://www.winzip.com}.}
\item{Using an \texttt{ansi-c} compiler (we recommend
\htmladdnormallink{Bloodshed
  Dev-C++}{http://www.bloodshed.net/dev/devcpp.html} - easy to install
and use (\ref{instblood})), the
\htmladdnormallink{McStas}{http://mcstas.risoe.dk} package can be
compiled using the \texttt{build.bat} script of the
\texttt{mcstas-1.7} directory you just unpacked. Follow the on screen
instructions.} 
\item{When the build has been done (e.g. \texttt{mcstas.exe}
has been produced), proceed to install (Section \ref{winbin}).}
\end{itemize}
\subsection{Unix build}
(NOTE: Taken almost directly from original INSTALL document - should
be customized before release!!!)\\
Prerequisites: Installed Perl, Tk and perl-tk packages.\\
For plotting, see Section \ref{plotting}.\\\ \\
McStas uses autoconf to detect the system configuration and create the
proper Makefiles needed for compilation. On Unix-like systems, you
should be able to compile and install McStas using the following steps:
\begin{enumerate}
\item{Unpack the sources to somewhere convenient and change to the
    source directory:\\\ \\

  gunzip -c mcstas-1.7-src.tar.gz | tar xf -\\
  cd mcstas-1.7/}

\item{Configure and compile McStas:\\\ \\

  ./configure\\
  make}
\item{Install McStas:\\\ \\

  make install}
\end{enumerate}
You should now be able to use McStas. For some examples to try, see the
examples/ directory.

The installation of McStas in step 3 by default installs in the
/usr/local/ directory, which on most systems requires superuser (root)
privileges. To install in another directory, use the --prefix= option to
configure in step 2. For example,

  ./configure --prefix=/home/joe

will install the McStas programs in /home/joe/bin/ and the library files
needed by McStas in /home/joe/lib/mcstas/.

In case ./configure makes an incorrect guess, some environment variables
can be set to override the defaults:

 - The CC environment variable may be set to the name of the C compiler
   to use (this must be an ANSI C compiler). This will also be used for
   the automatic compilation of McStas simulations in mcgui and mcrun.
 - CFLAGS may be set to any options needed by the compiler (eg. for
   optimization or ANSI C conformance). Also used by mcgui/mcrun.
 - PERL may be set to the path of the Perl interpreter to use.

To use these options, set the variables before running ./configure. Eg.

    setenv PERL /pub/bin/perl5\\
    ./configure

It may be necessary to remove configure's cache of old choices first:

    rm -f config.cache

If you experience any problems, or have some questions or ideas
concerning McStas, please contact \htmladdnormallink{peter.willendup@risoe.dk}{mailto:peter.willendup@risoe.dk}.

You should try to make sure that the directory containing the McStas
binaries (mcstas, gscan, mcdisplay, etc.) is contained in the PATH
environment variable. The default directory is /usr/local/bin, which is
usually, but not always, included in PATH. Alternatively, you can
reference the McStas programs using the full path name, ie.

  /usr/local/bin/mcstas my.instr\\
  perl /usr/local/bin/mcrun -N10 -n1e5 mysim -f output ARG=42\\
  perl /usr/local/bin/mcdisplay --multi mysim ARG=42\\

This may also be necessary for the front-end programs if the install
procedure could not determine the location of the perl interpreter on
your system.

If McStas is installed properly, it should be able to find the files it
needs automatically. If not, you should set the MCSTAS environment
variable to the directory containing the runtime files "mcstas-r.c" and
"mcstas-r.h" and the standard components (*.comp). Use one of

  MCSTAS=/usr/local/lib/mcstas; export MCSTAS     \# sh, bash\\
  setenv MCSTAS /usr/local/lib/mcstas             \# csh, tcsh\\

The PGPLOT library, which is used by the mcdisplay frontend, needs the
PGPLOT\_DIR environment variable to be set to the directory containing
PGPLOT, eg.

  PGPLOT\_DIR=/usr/lib/pgplot; export PGPLOT\_DIR   \# sh, bash\\
  setenv PGPLOT\_DIR /usr/lib/pgplot               \# csh, tcsh\\

See the PGPLOT documentation for details.

\section{Binary install, Linux}
\label{linbin}
Start from 'make install' in Section \ref{src}.
\section{Binary install, Windows}
\label{winbin}
\begin{itemize}
\item{Start by unpacking the \texttt{mcstas-1.7-i686-unknown-Win32.zip} package using
e.g. \htmladdnormallink{Winzip}{http://www.winzip.com}.}
\item{Execute the \texttt{install.bat} installation script. Follow the
  on screen instructions.}
\item{Set the required (see output of \texttt{install.bat}) environment variables using
\\\ \\
'Start/Settings/Control Panel/System/Advanced/Environment
Variables'\\\ \\}
\item{To get a fully working McStas environment, you must also install
    an \texttt{ansi-c} compiler, for instance BloodShed Dev-C++
    (Section \ref{instblood})}
\item{To get a functional graphical user
    interface and plotting facilities, you must also install the following
    packages:}
  \begin{itemize}
    \item{ActivePerl and ActiveTcl for GUI (see Section \ref{perltkwin})}
    \item{Matlab or Scilab for plotting (see Section \ref{plotting})}
  \end{itemize}
\end{itemize}
\section{Installing support Apps}
\subsection{Bloodshed Dev-C++ (Win32)}
\label{instblood}
To install Bloodshed Dev-C++, download the installer package from
\\\
\\\htmladdnormallink{http://www.bloodshed.net/dev/devcpp.html}{http://www.bloodshed.net/dev/devcpp.html}.\\\
\\
When installed, add the \texttt{bin} subdirectory to your path
using\\\ \\
'Start/Settings/Control Panel/System/Advanced/Environment Variables'.
\subsection{ActivePerl + ActiveTcl (Win32)}
\label{perltkwin}
\begin{itemize}
\item{Get and install the ActivePerl package from\\
    \htmladdnormallink{http://www.activestate.com/Products/Download/Register.plex?id=ActivePerl}{http://www.activestate.com/Products/Download/Register.plex?id=ActivePerl}
    \\(Registration
    not required)}
\item{Get and install the ActiveTcl package from\\
    \htmladdnormallink{http://www.activestate.com/Products/Download/Register.plex?id=ActiveTcl}{http://www.activestate.com/Products/Download/Register.plex?id=ActiveTcl}
    \\(Registration
    not required)}
\end{itemize}
\subsection{Plotting backends (All platforms)}
For plotting with McStas, different support packages can be used:
\begin{itemize}
\item{\htmladdnormallink{PGPLOT}{http://www.astro.caltech.edu/~tjp/pgplot/}/\htmladdnormallink{PDL}{http://pdl.perl.org}/\htmladdnormallink{pgperl}{http://www.ast.cam.ac.uk/AAO/local/www/kgb/pgperl/} (Unix only) - Binary builds of the packages
    exist for various Linux distributions (for instance
    \htmladdnormallink{Debian}{http://www.debian.org} comes with
    prebuilt versions). Prebuilt versions also exist for some commercial Unix'es. 
    Refer to distributor/vendor for documentation. The packages can also be
    built from source using some (in many cases much) effort.}
\item{Matlab (Some Unix/Win32) - refer to
    \htmladdnormallink{http://www.mathworks.com}{http://www.mathworks.com}. Matlab licenses are rather costly, but discount programmes for university and research departments exist.}
\item{Scilab (Unix/Win32/Mac...) - a free 'Matlab-like' package, available from
    \htmladdnormallink{http://www-rocq.inria.fr/scilab/}{http://www-rocq.inria.fr/scilab/}. McStas also requires the Plolib library from \htmladdnormallink{http://www.dma.utc.fr/~mottelet/myplot.html}{http://www.dma.utc.fr/~mottelet/myplot.html}.}
\end{itemize}
\label{plotting}
\end{document}