\section{Virtual\_output: Saving the first part of a split simulation}
\label{virtual_output}
\index{Sources!Virtual source, recording neutron events}

\component{Virtual\_output}{System}{filename}{buffer-size, type}{}

The component {\bf Virtual\_output} stores the neutron ray parameters
at the end of the first part of a split simulation. The idea is to let the
next part of the split simulation be performed by another instrument file,
which reads the stored neutron ray
parameters by the component {\bf Virtual\_input}.

All neutron ray parameters are saved to the output file, which is by default
of ``text'' type, but can also assume the binary formats
``float'' or ``double''. The storing of neutron rays continues until the
specified number of simulations have been performed.

\verb+buffer-size+ may be used to limit the size of the output file, but
absolute intentities are then likely to be wrong.
Exept when using MPI, we recommend to use the default value of zero, saving all neutron rays.
The size of the file is then controlled indirectly with the general $ncounts$ parameter.
