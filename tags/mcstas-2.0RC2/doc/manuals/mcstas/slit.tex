\section{Slit: A beam defining diaphragm}
\label{slit}
\index{Optics!Slit}

\component{Slit}{System}{$x_{\rm min}$, $x_{\rm max}$, $y_{\rm min}$, $y_{\rm max}$}{$r$, $p_{\rm cut}$}{}

The component {\bf Slit} is a very simple construction.
It sets up an opening at $z=0$, and propagates the neutrons
onto this plane (by the kernel call PROP\_Z0).
Neutrons within the slit opening are unaffected,
while all other neutrons
are discarded by the kernel call ABSORB.

By using {\rm Slit}, some neutrons contributing to the background
in a real experiment will be neglected.
These are the ones that scatter off the inner side
of the slit, penetrates the slit material,
or clear the outer edges of the slit.

The input parameters of {\bf Slit} are the four coordinates,
$(x_{\rm min}, x_{\rm max}, y_{\rm min}, y_{\rm max})$
defining the opening of the rectangle, or the radius $r$ of
a circular opening, depending on which parameters are specified.

The slit component can also be used to discard insignificant 
({\em i.e.}\ very low weight)
neutron rays, that in some simulations may be very abundant and therefore
time consuming. If the optional parameter $p_{\rm cut}$ is set, all
neutron rays with $p<p_{\rm cut}$ are ABSORB'ed.
This use is recommended in connection with {\bf Virtual\_output}.


