\section{TOF\_monitor: The time-of-flight monitor}
\component{TOF\_monitor}{System}{$x_{\rm min}$, $x_{\rm max}$, $y_{\rm min}$, $y_{\rm max}$, $n_{\rm chan}$, $t_0$, $t_1$, filename}{$\Delta t$}{}
\index{Monitors!Time-of-flight monitor}

The component {\bf TOF\_monitor} has a rectangular opening
in the $(x,y)$ plane, given by the $x$ and $y$ parameters,
like for {\bf Slit}.
The neutron ray is propagated to the plane of the monitor
by the kernel call PROP\_Z0.
A neutron ray is counted if it passes within the rectangular opening
given by the $x$ and $y$ limits.

Special about {\bf TOF\_monitor} is that it is sensitive to
the arrival time, $t$, of the neutron ray.
Like in a real time-of-flight detector, the time dimension is
binned into small time intervals.
Hence this monitor maintains a one-dimensional histogram of counts.
The $n_{\rm chan}$ time intervals begin at $t_0$ and
end at $t_1$ (alternatively, the interval length is specified by $\Delta t$).
As usual in time-of-flight analysis, all times are given in units of $\mu$s.

The output parameters from {\bf TOF\_monitor} are the three count numbers,
$N, I$, and $M_2$ for the total counts in the monitor.
In addition, a file, \verb+filename+, is produced with a list of
the same three data divided in different TOF bins.
This file can be read and plotted by the {\rm mcplot} tool; see the
System Manual.

