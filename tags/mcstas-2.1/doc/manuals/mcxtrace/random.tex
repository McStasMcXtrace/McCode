\chapter{Random numbers in \MCX}
\label{s:random}
\index{Monte Carlo method}

\section{Transformation of random numbers}
In order to perform the Monte Carlo choices, one needs to be able to
pick a random number from a given distribution. However, most
random number generators only give
uniform distributions over a certain interval.
We thus need to be able to transform between probability distributions,
and we here give a short explanation on how to do this.

Assume that we pick a random number, $x$, from a distribution $\phi(x)$.
We are now interested in the shape of the distribution, $\Psi(y)$, of the
transformed $y=f(x)$, assuming $f(x)$ is monotonous.
All random numbers lying in the interval $[x; x+dx]$
are transformed to lie within the interval $[y; y+f'(x)dx]$, so the
resulting distribution must be $\Psi(y) = \phi(x) / f'(x)$.

If the random number generator selects numbers uniformly in the interval
$[0; 1]$, we have $\phi(x) = 1$ (inside the interval; zero outside), and
we reach
\begin{equation}
\Psi(y) = \frac{1}{f'(x)} = \frac{d}{dy} f^{-1}(y) .
\end{equation}
By indefinite integration we reach
\begin{equation}
\label{e:randtrans}
\int \Psi(y) dy = f^{-1}(y) = x ,
\end{equation}
which is the essential formula for random number transformation, since we
in general know $\Psi(y)$ and like to determine the relation $y=f(x)$.
Let us illustrate with a few examples of transformations relevant for the
\MCX\ components.

\paragraph{The circle}
For finding a random point within the
circle of radius $R$, one would like to choose the polar angle, $\phi$,
from a uniform
distribution in $[0; 2\pi]$, giving $\Psi_\phi = 1/(2\pi)$.
and the radius from the (normalised) distribution $\Psi_r=2r/R^2$.

For the radial part,
eq.~(\ref{e:randtrans}) becomes $y/(2 \pi) = x$, whence
$\phi$ is found simply by multiplying a random number ($x$)
with $2\pi$.

For the radial part, the left side of eq.~(\ref{e:randtrans}), gives
$\int \Psi(r) dr = \int 2 r/R^2 dr = r^2/R^2$,
which from (\ref{e:randtrans}) should equal $x$.
Hence we reach the wanted transformation $r = R\sqrt{x}$.

\paragraph{The sphere}
For finding a random point on the surface of the unit sphere,
we need to determine the two angles, $(\theta, \phi)$.

$\Psi_\phi$ is chosen from a uniform distribution
in $[0; 2\pi]$, giving $\phi = 2\pi x$ as for the circle.

The probability distribution of $\theta$ should be
$\Psi_\theta=\sin(\theta)$ (for $\theta \in [0; \pi ]$),
whence by eq.~(\ref{e:randtrans}) $\theta=\cos^{-1}(x)$.

\paragraph{Exponential decay}
In a simple time-of-flight source, the neutron flux decays exponentially
after the initial activation at $t=0$. We thus want to pick an initial
neutron emission time from the normalised distribution
$\Psi(t) = \exp(-t/\tau) / \tau$.
Use of Eq.~(\ref{e:randtrans}) gives
$x = 1 - \exp(-t/\tau)$. For convenience we now use the random variable
$x_1 = 1-x$ (with the same distributions as $x$),
giving the simple expression $t = - \tau \ln (x_1)$.

\paragraph{Normal distributions}
The important normal distribution can not be reached as a simple
transformation of a uniform distribution.
In stead, we rely on a specific algorithm for selecting random
numbers with this distribution.

\section{Random generators}
\index{Monte Carlo method!Random number, Mersenne Twister}
Even though there is the possibility to use the system random generator, as well as the initial \MCX\ random generator, the default algorithm is the so-called "Mersenne Twister", by Makoto Matsumoto and Takuji Nishimura. See \\ \verb+http://www.math.sci.hiroshima-u.ac.jp/~m-mat/MT/emt.html+ for original source.

It is considered today to be by far the best random generator, which means that both its period is extremely large $2^{19937}-1$, and cross-correlations are negligible, i.e distributions are homogeneous and independent up to 623 dimensions. It is also extremely fast.
